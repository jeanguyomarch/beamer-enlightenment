\documentclass[aspectratio=169]{beamer}
\usepackage[utf8]{inputenc}
\usepackage[T1]{fontenc}
\usepackage{lipsum}

\title[Short Title]{The Title}
\subtitle{Presentation Subtitle}
\date[\textbackslash{}today]{Date}
\author[Jean Guyomarc'h]{Jean Guyomarc'h\\\texttt{jean.guyomarch@gmail.com}}

\usetheme{enlightenment}

\begin{document}

\begin{frame}
\titlepage
\end{frame}


\begin{frame}{Frame Title}{Frame Subtitle}
   This slide contains itemizations and enumerations:
   \begin{itemize}
      \item look at this bullet!
      \item Oh, another one!
      \item That's magical, let's count to 3:
         \begin{enumerate}
            \item that's one;
               \begin{itemize}
                  \item one and a half;
                  \item almost two;
               \end{itemize}
            \item that's two;
            \item that's three;
               \begin{enumerate}
                  \item four?
                  \item no...
               \end{enumerate}
         \end{enumerate}
   \end{itemize}
\end{frame}

\begin{frame}{Lorem Ipsum}{dolor sit amet...}
   \lipsum[1]
\end{frame}


\begin{frame}[fragile]{Verbatim}
   \begin{verbatim}
      This is verbatim with \LaTeX{} commands
      that are well escaped and stuff...
   \end{verbatim}
\end{frame}

\begin{frame}{How about maths?}{Entropy}
   Let $H(X)$ be the entropy:
   \[%
      \boxed{%
         H(X) = -\sum_{i=1}^{n}{P(x_i)\log_bP(x_i)}
      }
   \]
   where:
   \begin{itemize}
      \item $X$ is a discrete random variable: $X=\{x_1, \cdots, x_n\}$;
      \item $b \in \mathbb{N}^{*+}$ defines the entropy base ($b=2$ for Shannon's);
      \item $P(X)$ is a probability mass function.
   \end{itemize}
\end{frame}

\begin{frame}{Figures}
   \begin{figure}[h!]\centering
      \includegraphics[scale=0.5]{img/e_logo}
      \caption{This is a figure}
   \end{figure}
\end{frame}

\begin{frame}[fragile]{Code Listings}
   \begin{lstlisting}
#define BARE_METAL_IN_CASE_YOU_DID_NOT_NOTICE
#include <stdio.h>

static volatile unsigned char *PIN = (void *)(0x02020000);

void // Yes, void!
main(void)
{
   const int my = 42;

   /* Directly access the hardware */
   *PIN = my;
   fprintf(stdout, "HW value is: %i\n", *PIN);
}
   \end{lstlisting}
\end{frame}


\end{document}
